\documentclass[11pt]{article}
\usepackage{amsmath,amssymb,color,float}
\usepackage{graphicx,psfrag,epsf}
\usepackage{natbib}


\setlength{\oddsidemargin}{.15in} 
\setlength{\textwidth}{6.25in}
\setlength{\topmargin}{-0.25in}
\setlength{\headheight}{-0.15in}
\setlength{\textheight}{8.9in} 

\linespread{1.25}


\title{ST 705 Linear models and variance components \\ 
        Homework problem set 3}


\begin{document}
\maketitle

\begin{enumerate}

\item Let $V$ be a convex subset of some vector space.  Recall that a function $f : V \to \mathbb{R}$ is said to be $convex$ if for every $x, y \in V$ and every $\lambda \in [0,1]$,
\[
f(\lambda x + (1-\lambda)y) \le \lambda f(x) + (1-\lambda) f(y).
\]
Show, by definition, that the sum of squared errors function
\[
Q(\beta) := \|Y - X\beta\|_{2}^{2}
\]
is convex.

\item Show that if $\text{rank}(BC) = \text{rank}(B)$, then $\text{col}(BC) = \text{col}(B)$, where col$(\cdot)$ denotes the column space.

\item Let $A \in \mathbb{R}^{n\times p}$ with rank$(A) = p$.  Further, suppose  $X \in \mathbb{R}^{n\times q}$ with $\text{col}(X) = \text{col}(A)$.  Show that there exists a unique matrix $S$ so that $X = AS$.

\item Show that the $R^{2}$ value for a simple linear regression can never achieve 1 if the observed data consists of repeated (different) observations of the response, $y$, at the same value of the predictor, $x$.

\item Suppose that the $m\times n$ matrix $A$ has the form 
\[
A = 
\begin{pmatrix}
A_{1} \\
A_{2} \\
\end{pmatrix}
\]
where $A_{1}$ is an $n\times n$ nonsingular matrix, and $m > n$.  Define $A^{+} := (A'A)^{-1}A'$, and prove that $\|A^{+}\|_{2} \le \|A_{1}^{-1}\|_{2}$.

\item Let $X \in \mathbb{R}^{n\times p}$, $u \in \mathbb{R}^{n}$, and $v \in \mathbb{R}^{p}$.  
\begin{enumerate}
\item Prove that
\[
|u'Xv| \le \bigg(\max_{1\le j\le p}\Big\{\sum_{i=1}^{n}|X_{i,j}|\Big\}\bigg)^{\frac{1}{2}} \bigg(\max_{1\le i\le n}\Big\{\sum_{j=1}^{p}|X_{i,j}|\Big\}\bigg)^{\frac{1}{2}} \cdot \|u\|_{2} \cdot \|v\|_{2}.
\]
\item Show that the Cauchy-Schwarz inequality is a special case of the inequality given in part (a).
\end{enumerate}



\end{enumerate}






\end{document}
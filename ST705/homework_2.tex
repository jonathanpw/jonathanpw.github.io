\documentclass[11pt]{article}
\usepackage{amsmath,amssymb,color,float}
\usepackage{graphicx,psfrag,epsf}
\usepackage{natbib}


\setlength{\oddsidemargin}{.15in} 
\setlength{\textwidth}{6.25in}
\setlength{\topmargin}{-0.25in}
\setlength{\headheight}{-0.15in}
\setlength{\textheight}{8.9in} 

\linespread{1.25}


\title{ST 705 Linear models and variance components \\ 
        Homework problem set 2}


\begin{document}
\maketitle

\begin{enumerate}

\item Monahan exercise A.50.

\item Let $A \in \mathbb{R}^{p\times p}$ be symmetric.  Use the spectral decomposition of $A$ to show that 
\[
\sup_{x\in\mathbb{R}^{p}\setminus\{0\}} \frac{x'Ax}{x'x} = \lambda_{\max},
\]
where $\lambda_{\max}$ is the largest eigenvalue of $A$.  Observe that this is a special case of the Courant-Fischer theorem (see \verb1https://en.wikipedia.org/wiki/Min-max_theorem1).

\item Construct an $n\times n$ matrix $A$ such that $\lambda_{\max}(A) \ne \sup\limits_{v\ne0}\big\{\frac{v'Av}{v'v}\big\}$, where $\lambda_{\max}(\cdot)$ denotes the maximum eigenvalue of its argument.  Why does your counter example not violate the Courant-Fischer theorem?

\item Let $A$ be a positive definite matrix, and show that 
\[
\text{tr}(I - A^{-1}) \le \log\det(A) \le \text{tr}(A - I).
\]

\item Let $A$ be an $m\times n$ matrix with rank $m$.  Prove that there exists an $n\times m$ matrix $B$ such that $AB = I_{m}$.

\item For matrices $A \in \mathbb{R}^{p\times q}$, the \textit{spectral} norm is defined as,
\[
\|A\|_{2} := \sqrt{\sup_{x\ne0}\frac{x'A'Ax}{x'x}}.
\]
Further, the eigenvalues of $A'A$ are the squares of the \textit{singular values} of $A$, so sometimes the definition of the spectral norm is expressed as
 \[
\|A\|_{2} := \sigma_{\max}(A),
\]
where $\sigma_{\max}$ denotes the largest singular value of $A$. 
\begin{enumerate}
\item Verify that the spectral norm is a norm.  Recall that a norm must satisfy the following axioms for any $A,B,C \in \mathbb{R}^{p\times q}$ and any $\alpha \in \mathbb{R}$.
\begin{enumerate}
\item $\|\alpha A\| = |\alpha|\|A\|$
\item $\|A + B\| \le \|A\| + \|B\|$
\item $\|A\| \ge 0$ with equality if and only if $A = 0$.
\end{enumerate}
\item Show that the spectral norm is sub-multiplicative for square matrices.  That is, for $A,B \in \mathbb{R}^{p\times p}$, $\|AB\|_{2} \le \|A\|_{2}\|B\|_{2}$.  
\end{enumerate}

\end{enumerate}






\end{document}
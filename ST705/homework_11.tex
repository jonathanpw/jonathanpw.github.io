\documentclass[11pt]{article}
\usepackage{amsmath,amssymb,color,float}
\usepackage{graphicx,psfrag,epsf}
\usepackage{natbib}


\setlength{\oddsidemargin}{.15in} 
\setlength{\textwidth}{6.25in}
\setlength{\topmargin}{-0.25in}
\setlength{\headheight}{-0.15in}
\setlength{\textheight}{8.9in} 

\linespread{1.25}


\title{ST 705 Linear models and variance components \\ 
        Homework problem set 11}


\begin{document}
\maketitle

\begin{enumerate}

\item Monahan exercise 5.10

\item Monahan exercise 5.12

\item Monahan exercise 5.14

\item Monahan exercise 5.16

\item Monahan exercise 5.19

\item Monahan exercise 5.23.  This is the premise for imputation in Gaussian processes regression.  Gaussian processes regression is a common approach to modeling data arising from stochastic processes or from spatial processes. 

\item Monahan exercise 5.27

\end{enumerate}






\end{document}
\documentclass[11pt]{article}
\usepackage{amsmath,amssymb,color,float}
\usepackage{graphicx,psfrag,epsf}
\usepackage{natbib}


\setlength{\oddsidemargin}{.15in} 
\setlength{\textwidth}{6.25in}
\setlength{\topmargin}{-0.25in}
\setlength{\headheight}{-0.15in}
\setlength{\textheight}{8.9in} 

\linespread{1.25}


\title{ST 705 Linear models and variance components \\ 
        Homework problem set 4}


\begin{document}
\maketitle

\begin{enumerate}

\item Monahan exercise A.34.

\item Monahan exercise A.35.

\item The defining property of a projection matrix $A$ is that $A^{2} = A$ (recall the definition of the square of a matrix from your linear algebra course).  Establish the following facts.
\begin{enumerate}
\item If $A$ is a projection matrix, then all of its eigenvalues are either zero or one. 
\item If $A \in \mathbb{R}^{p\times p}$ is a projection and symmetric (i.e., an orthogonal projection matrix), then for every vector $v$ the projection $Av$ is orthogonal to $v - Av$.
\item $\text{tr}(A + B) = \text{tr}(A) + \text{tr}(B)$.
\item $\text{tr}(AB) = \text{tr}(BA)$.
\end{enumerate}

\item In lecture, we proved a lemma that $(X'X)^{g}X'$ is a generalized inverse of $X$.
\begin{enumerate}
\item Verify that $X(X'X)^{g}$ is a generalized inverse of $X'$.
\item We proved that $P_{X} := X(X'X)^{g}X'$ is the unique projection onto column($X$).  Is $(X'X)^{g}X'$ the unique generalized inverse of $X$?  
\end{enumerate}

\item Let $X = QR$ where $Q$ has orthonormal columns.  Prove that if $\text{rank}(X) = \text{rank}(Q)$, then $P_{X} = QQ'$.

\item Let $A \in \mathbb{R}^{n\times p}$.
\begin{enumerate}
\item Prove that if $A^{g}$ is a generalized inverse of $A$ (i.e., only satisfying $AA^{g}A = A$), then $(A^{g})'$ is a generalized inverse of $A'$.  Conclude from this fact that $P_{X} := X(X'X)^{g}X'$ is symmetric.
\item Prove the existence \textbf{and} uniqueness of the Moore-Penrose generalized inverse, usually denoted $A^{+}$, of $A$.
\item Show that if $A$ has full column rank, then $A^{+} = (A'A)^{-1}A'$.  
\item Show that if $A$ has full row rank, then $A^{+} = A'(AA')^{-1}$.
\end{enumerate}



\end{enumerate}






\end{document}